\documentclass[11pt,a4paper]{moderncv}        % possible options include font size ('10pt', '11pt' and '12pt'), paper size ('a4paper', 'letterpaper', 'a5paper', 'legalpaper', 'executivepaper' and 'landscape') and font family ('sans' and 'roman')
\usepackage{xcolor, soul}
% moderncv themes
\moderncvstyle{banking}                            % style options are 'casual' (default), 'classic', 'banking', 'oldstyle' and 'fancy'
\moderncvcolor{black}                               % color options 'black', 'blue' (default), 'burgundy', 'green', 'grey', 'orange', 'purple' and 'red'
%\renewcommand{\familydefault}{\sfdefault}         % to set the default font; use '\sfdefault' for the default sans serif font, '\rmdefault' for the default roman one, or any tex font name
% \nopagenumbers{}                                  % uncomment to suppress automatic page numbering for CVs longer than one page
 
% character encoding
%\usepackage[utf8]{inputenc}                       % if you are not using xelatex ou lualatex, replace by the encoding you are using
%\usepackage{CJKutf8}                              % if you need to use CJK to typeset your resume in Chinese, Japanese or Korean

% adjust the page margins
\usepackage[scale=0.85]{geometry}
%\setlength{\hintscolumnwidth}{3cm}                % if you want to change the width of the column with the dates
%\setlength{\makecvheadnamewidth}{10cm}            % for the 'classic' style, if you want to force the width allocated to your name and avoid line breaks. be careful though, the length is normally calculated to avoid any overlap with your personal info; use this at your own typographical risks...

% personal data
\name{Yufeng}{Liu}
\title{Curriculum Vitae}   
\AtEndPreamble{
  \hypersetup{
    colorlinks=true,
    linkcolor=black,
    urlcolor=cyan
}}
% \address{Harbin Institute of Technology(Shenzhen)}{Shenzhen}{China}% optional, remove / comment the line if not wanted; the "postcode city" and "country" arguments can be omitted or provided empty
% \phone[mobile]{+86~13029155000}                   % optional, remove / comment the line if not wanted; the optional "type" of the phone can be "mobile" (default), "fixed" or "fax"
\email{raymond.lau.lyf@gmail.com}                              % optional, remove / comment the line if not wanted
\extrainfo{\emailsymbol\emaillink{yufeng004@e.ntu.edu.sg}}
\homepage{raymond-lau-lyf.github.io}                         % optional, remove / comment the line if not wanted

\renewcommand*{\bibliographyitemlabel}{[\arabic{enumiv}]}

\begin{document}
%\begin{CJK*}{UTF8}{gbsn}                          % to typeset your resume in Chinese using CJK

%-----       resume       ---------------------------------------------------------
\makecvtitle 

\section{Education}
\cventry{Sept.2020--Jun.2024}{B.Eng. in Automation  GPA:87/100 IELTS 6.5}{Harbin Institute of Technology(Shenzhen)}{Shenzhen,China}{}{} % arguments 3 to 6 can be left empty
\cventry{Aug.2024--present}{M.Sc. in Computer Control and Automation }{Nanyang Technological University}{Singapore}{}{} % arguments 3 to 6 can be left empty


\section{Experience}

\cventry{Oct.2021-Jun.2024}{}{Multi sensor SLAM in complex environments.}{\href{https://www.nrs-lab.com/}{nROS-Lab},HITsz}{}%
{
\begin{itemize}%
\item Participated in the implementation and experiment of an Edge-Based Monocular Thermal-Inertial Odometry \hyperref[sec:Publications]{[publication]}.
\begin{itemize}%
\hypersetup{urlcolor=black}
\item Developed a simulation system in \href{https://gazebosim.org/api/gazebo/2.10/index.html}{Ignition Gazebo} for SLAM in complex extreme environments.
\item Deployed the \hyperref[sec:Publications]{algorithm} in the real world and conducted experiments in the real world and datasets.
\item Skilled in thermal image processing.
\item Familiar with the system framework of VIOs like VINS-Mono, ORB-SLAM3, etc.
\end{itemize}
\item Proposed a SLAM framework that fuses thermal camera, LiDAR, and IMU.
\begin{itemize}%
\item Designed a novel multi-sensor SLAM framework specially designed for sensor-degraded scenes. 
\item Skilled in multi-sensor calibration.
\item Skilled in approaches to perform multi-sensor time synchronization.
\item Familiar with common multi-sensor SLAM frameworks like LVI-SAM, R2Live, R3Live, FAST-LIVO, etc.
\item (This project is my Final Year Project \& Dissertation, which won the HITsz Outstanding Final Year Project \& Dissertation Award)
\item (Related journal publication is expected to be released in the near future.)
\end{itemize}
\item Participated in the implementation of a SLAM system integrated planning and dynamic obstacle avoidance.
\begin{itemize}%
\item Applied deep-learning method for target detection to optimize the LiDAR odometry.
\item Designed shared memory method for pointcloud data acceleration.
\end{itemize}
\end{itemize}
}

\cventry{Oct.2022--Sept.2023}{}{Teleoperated robot equipped with a VR remote-controlled gimbal system.}{\href{https://www.nrs-lab.com/}{nROS-Lab},HITsz}{}%
{
\begin{itemize}%
\item Designed a two-axis gimbal with sensors for mobile robots:
\begin{itemize}%
\item Designed the 3D model and implemented real-time embedded control.
\item Developed a framework for human-computer interaction, as well as a VR application.
\item Deployed Multi-sensor SLAM algorithm on the gimbal.  
\end{itemize}
\end{itemize}
}

\cventry{Oct.2020--Aug.2022}{}{Team leader of Sentry Robot Group in \href{http://www.robomaster.com/en-US}{RoboMaster} competition}{Critical-HIT robot team,HITsz}{}%
{
\begin{itemize}%
\item Led the Sentry Robot Group in HITsz Critical-HIT RoboMaster Team.
\begin{itemize}
\item Designed a fully automatic inspection and combat-integrated robot.
\item Coordinated task allocation and fostered collaboration among team members as team leader.
\item Responsible for embedded. 
\item Developed target aiming algorithm framework, including target detection tracking.
\end{itemize}
\end{itemize}
}

\cventry{May.2022--Dec.2022}{}{Underwater grab robot control and navigation}{Lujian Technology Ltd. Co.,Shenzhen}{}%
{
\begin{itemize}%
    \item Participated in the design of an underwater robot  
    \item Responsible for visual-inertial odometry and planning in underwater environments.
    \item Responsible for embedded motion control.
    \item Achieved a learning-based underwater target detection.
\end{itemize}
}
\cventry{}{}{More detailed experiences can be explored at \href{https://raymond-lau-lyf.github.io/projects/}{Website}.}{}{}%
{}

\section{Skills}
\cvitem{}{Programming: C++, C, Python, MATLAB}
\cvitem{}{Software \& tools: ROS, OpenCV, Gazebo, PCL, GTSAM, Ceres, Git, PyTorch, LaTeX, Qt Creator, Unity }
\cvitem{}{Hardware: STM32, SolidWorks}


\section{Publications}
\label{sec:Publications}
\cvitem{}{[1] Yu Wang, Haoyao Chen, \textbf{Yufeng Liu}, and Shiwu Zhang. Edge-based monocular thermal-inertial odometry in visually degraded environments. IEEE Robotics and Automation Letters(RA-L), 8(4):2078-2085, 2023. \href{https://ieeexplore.ieee.org/document/10048516}{[link]} \href{https://arxiv.org/abs/2210.10033}{[arxiv]}}
% \cvitem{}{[2] Edge-Feature-Based, Degradation-Aware Thermal-Inertial-LiDAR Odometry. (Submissions in progress as Co-First Author)}
% \cvitem{}{[3] e2vio: Enhanced Edge-based Visual-Inertial Odometry. (Submissions in progress as Co-First Author)}


\section{Awards}
\begin{itemize}%
    \item Outstanding Final Year Project \& Dissertation Award - Top2\% of HITsz \hfill 2024
    \item First Prize of 2022 RoboMaster University Championship\hfill 2022
    \item Silver Prize of 13th  Challenge Cup\hfill 2022
    \item First Prize of 2021 RoboMaster University Championship\hfill 2021
    \item Third Prize of ChinaUndergraduate Mathematical Contest in Modelling\hfill 2021
    \item First Place among all students of Competition of the HITsz Robot Design and Practice Course\hfill 2020
\end{itemize}

% \clearpage\end{CJK*}                              % if you are typesetting your resume in Chinese using CJK; the \clearpage is required for fancyhdr to work correctly with CJK, though it kills the page numbering by making \lastpage undefined
\end{document}

%% end of file `template.tex'.
